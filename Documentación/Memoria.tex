\documentclass[11pt]{article}
\usepackage{natbib}
\usepackage{url}
\usepackage[utf8]{inputenc} % Codificación
\usepackage{amsmath}
\usepackage{graphicx}
\graphicspath{{images/}} % Carpeta en la cual se van a buscar las imagenes
\usepackage{subfigure}	% Permite la Inclusión de subfiguras
%\usepackage{parskip} % Borrar identación de parrafos.
\setlength{\parskip}{3mm} % Longitud del espaciado entre parrafos
\usepackage[hidelinks]{hyperref} % Referencias (links)
\usepackage{fancyhdr}
\usepackage{vmargin}
\setpapersize{A4} % Formato del papel - A4
\setmarginsrb{3 cm}{2.5 cm}{3 cm}{2.5 cm}{1 cm}{1.5 cm}{1 cm}{1.5 cm} % Margenes
\usepackage{paralist} % Permite un mayor control sobre las listas
\usepackage{textcomp,marvosym,pifont} % Generación de símbolos especiales
\usepackage[usenames,dvipsnames,svgnames,x11names,table]{xcolor}

%%%%%%%%%%%%%%%%%%%%%%%%%%%%%%%%%%%%%%%%%%%%%%%%%%%%%%%%%%%%%%%%%%%%%%%%%%%%%%%%%%%%%%%%%
% Complemento para insertar código en la memoria:
% A partir de 'Listados de código cómodos y resultones con listings' de David Villa
% http://crysol.org/es/node/909

\usepackage{color}
\definecolor{gray97}{gray}{.97}
\definecolor{gray75}{gray}{.75}
\definecolor{gray45}{gray}{.45}

\usepackage{listings}
\lstset{ frame=Ltb,
	framerule=0pt,
	aboveskip=0.5cm,
	framextopmargin=3pt,
	framexbottommargin=3pt,
	framexleftmargin=0.4cm,
	framesep=0pt,
	rulesep=.4pt,
	backgroundcolor=\color{gray97},
	rulesepcolor=\color{black},
	texcl=true,
	%
	stringstyle=\ttfamily,
	showstringspaces = false,
	basicstyle=\small\ttfamily,
	commentstyle=\color{gray45},
	keywordstyle=\bfseries,
	%
	numbers=left,
	numbersep=15pt,
	numberstyle=\tiny,
	numberfirstline = false,
	breaklines=true,
}

% minimizar fragmentado de listados
\lstnewenvironment{listing}[1][]
{\lstset{#1}\pagebreak[0]}{\pagebreak[0]}

\lstdefinestyle{consola}
{basicstyle=\scriptsize\bf\ttfamily,
	backgroundcolor=\color{gray75},
}

\lstdefinestyle{C}
{language=C,
}
%%%%%%%%%%%%%%%%%%%%%%%%%%%%%%%%%%%%%%%%%%%%%%%%%%%%%%%%%%%%%%%%%%%%%%%%%%%%%%%%%%%%%%%%%

\usepackage{float} % Permite usar H en las figuras, de manera que se coloquen en la posición exacta en la que están en el código.


% Añade un comando para crear indicaciones de pulsación de teclas
\usepackage{tikz} % Paquete especializado en gráficos
\usetikzlibrary{shadows} % Necesario para poder crear nuevo comando de indicación de pulsación de tecla.
\newcommand*\tecla[1]{%   
	\tikz[baseline=(key.base)]
	\node[%
	draw,
	fill=white,
	drop shadow={shadow xshift=0.25ex,shadow yshift=-0.25ex,fill=black,opacity=0.75},
	rectangle,
	rounded corners=2pt,
	inner sep=1pt,
	line width=0.5pt,
	font=\scriptsize\sffamily
	](key) {#1\strut}
	;
}

\newif\ifspanish % Condicional que permite seleccionar el lenguage.
\spanishtrue


%%%%%%%%%%%%%%%%%%%%%%%%%%%%%%%%%%%%%%%%%%%%%%%%%%%%%%%%%%%%%%%%%%%%%%%%%%%%%%%%%%%%%%%%%
%%%%%%%%%					Principales variables del documento					%%%%%%%%%

\title{Práctica NCuerpos.}							% Titulo
\author{Marcos López Sobrino y Alberto Salas Seguín.}							% Autor
\date{\today}											% Fecha
\newcommand{\subject}{Computadores Avanzados.}						% Asignatura
\newcommand{\course}{4º Grado en Ingeniería Informática.}		% Curso

%\spanishfalse	% Descomentar esta línea si el trabajo está en inglés

%%%%%%%%%%%%%%%%%%%%%%%%%%%%%%%%%%%%%%%%%%%%%%%%%%%%%%%%%%%%%%%%%%%%%%%%%%%%%%%%%%%%%%%%%

\ifspanish
	\usepackage[spanish]{babel} % Paquete de español
	\newcommand{\dateText}{Fecha:}
	\renewcommand{\lstlistingname}{Listado} % Renombrar listados para que aparezcan en español.
	% Algoritmos
	\usepackage[ruled,vlined,spanish]{algorithm2e} % Permite pseudocódigos. NECESARIO INSTALAR texlive-science (sudo apt-get install texlive-science)
\else
	\usepackage[english]{babel} % Paquete de inglés
	\newcommand{\dateText}{Date:}
	% Algoritmos
	\usepackage[ruled,vlined,english]{algorithm2e}
\fi

\makeatletter
\let\thetitle\@title
\let\theauthor\@author
\let\thedate\@date
\makeatother

\pagestyle{fancy}
\fancyhf{}
\rhead{\theauthor}
\lhead{\thetitle}
\cfoot{\thepage}

\begin{document}
\begin{titlepage}
	\centering
    \includegraphics[scale = 0.25]{esilogo.png}\\[1.0 cm]	% Logo de la universidad
    \textsc{\LARGE Universidad de Castilla-La Mancha}\\[0.5 cm]	% Nombre de la universidad
    \textsc{\LARGE Escuela Superior de Informática}\\[2.0 cm]
	\textsc{\Large \textbf{\subject}}\\[0.5 cm]				% Asignatura
	\textsc{\large \course}\\[0.5 cm]						% Curso
	\rule{\linewidth}{0.2 mm} \\[0.4 cm]
	{ \huge \bfseries \thetitle}\\
	\rule{\linewidth}{0.2 mm} \\[1.5 cm]
	
	\begin{minipage}{0.4\textwidth}
		\begin{flushleft} \large
			\emph{Autor:}\\
			\textbf{\theauthor}
			\end{flushleft}
			\end{minipage}~
			\begin{minipage}{0.4\textwidth}
			\begin{flushright} \large
			\emph{\dateText} \\
			\thedate
		\end{flushright}
	\end{minipage}\\[1.5 cm]
 
	\vfill
	
\end{titlepage}

\tableofcontents
\newpage

\section{Instrucciones para la ejecución de los programas.}
Para la compilación y ejecución de los programas, se ha decidido realizar un \textit{Makefile}, donde tenemos las siguientes instrucciones:
\begin{lstlisting}[style=C, numbers=none]
all: compile-par-rapido run-par-rapido
\end{lstlisting}
Mediante esta instrucción, primero se compila el programa paralelo rápido y seguidamente, una vez compilado, se ejecuta dicho programa. \\

\begin{lstlisting}[style=C, numbers=none]
compile-sec:
	gcc NCuerposSecuencial.c -o NCuerposSecuencial -lm -Wall
\end{lstlisting}
Con esta instrucción, se compila el programa secuencial.\\

\begin{lstlisting}[style=C, numbers=none]
run-sec:
	./NCuerposSecuencial
\end{lstlisting}
A través de esta instrucción, el programa secuencial es ejecutado. \\

\begin{lstlisting}[style=C, numbers=none]
compile-par:
	mpicc NCuerposParalelo.c -o NCuerposParalelo -lm -Wall
\end{lstlisting}
Mediante instrucción se compila el programa paralelo básico. \\

Siguiendo la transparencia 30 del documento, si solo interesa el tiempo el tiempo de ejecución, se ha incluido la directiva de compilación condicional \textbf{\textit{\# ifndef NO\_SAL}} en el programa, de modo que para únicamente obtener el tiempo de ejecución, en el terminal tenemos que introducir la instrucción \textbf{\textit{make compile-par-nosal}}. Esta instrucción es:

\begin{lstlisting}[style=C, numbers=none]
compile-par-nosal:
   mpicc NCuerposParalelo.c -o NCuerposParalelo -lm -Wall -D NO_SAL
\end{lstlisting}
La ejecución del programa paralelo básico se hace con la siguiente instrucción.
\begin{lstlisting}[style=C, numbers=none]
run-par:
	mpirun -np 2 NCuerposParalelo
\end{lstlisting}
\newpage

\begin{lstlisting}[style=C, numbers=none]
compile-par-rapido:
	mpicc NCuerposParalelo_AlgoritmoRapido.c -o NCuerposParalelo_AlgoritmoRapido -lm -Wall
\end{lstlisting}

Esta instrucción es utilizada para la compilación del programa paralelo rápido. \\

Al igual que antes, si únicamente nos interesa el tiempo de ejecución del programa, la instrucción a utilizar es la siguiente:
\begin{lstlisting}[style=C, numbers=none]
compile-par-rapido-nosal:
	mpicc NCuerposParalelo_AlgoritmoRapido.c -o NCuerposParalelo_AlgoritmoRapido -lm -Wall -D NO_SAL
\end{lstlisting}

Con la instrucción \textit{run-par-rapido} se ejecuta el programa paralelo rápido.
\begin{lstlisting}[style=C,numbers=none]
run-par-rapido:
	mpirun -np 2 NCuerposParalelo_AlgoritmoRapido
\end{lstlisting}


\newpage
\section{Explicaciones de diseño.}
En primer lugar, se han implementado 3 estructuras:
\begin{itemize}
	\item \textbf{struct Datos}. Dicha estructura contiene las variables \textbf{n, tp, k, delta, u} necesarias para el algoritmo.
	
	\item \textbf{struct Masas}. En esta estructura se encuentra el \textbf{id} y la masa (\textbf{m}) de cada cuerpo.
	
	\item \textbf{struct Coord}. Donde almacenamos el \textbf{id}, la posición \textbf{x} y la posición \textbf{y} de cada uno de los cuerpos.
\end{itemize}

Para poder trabajar con estas estructuras se ha utilizado \textbf{\textit{MPI\_Datatype}}.
\begin{lstlisting}[style=C,numbers=none]
MPI_Datatype MPI_Datos;
MPI_Datatype MPI_Masas;
MPI_Datatype MPI_Coord;
MPI_Datatype MPI_CNCR;
\end{lstlisting}
\textbf{\textit{MPI\_Datatype MPI\_CNCR}} se ha utilizado para la implementación del algoritmo con distribución cíclica como se dice en el documento a partir de la diapositiva 44. De este modo, el proceso 0 va a almacenar las posiciones de todos los cuerpos y se las va a enviar a todos los procesos mediante la primitiva \textbf{\textit{MPI\_Scatter}}. \\

Las operaciones que se han utilizado han sido las mencionadas en el enunciado y explicadas en el documento \textit{Introducción a MPI}. \\
\begin{itemize}
	\item Para el envío y recepción de las masas se ha utilizado la operación \textbf{\textit{MPI\_Bcast}}.
	\item Para el envío y recepción de las posiciones y velocidades desde el rank 0 a los demás, se ha optado por 						\textbf{\textit{MPI\_Scatter}} como hemos mencionado anteriormente.
	\item Para sincronizar a todos los procesos \textbf{\textit{MPI\_Barrier}}.
	\item Para que el proceso 0, pueda imprimir los valores de los cuerpos y que se produzca una salida ordenada, primero debe 			recibir todos los datos, que son las velocidades y aceleraciones de los cuerpos asignados a otros procesos. Para este fin, se 	ha usado \textbf{\textit{MPI\_Gather}}.
\end{itemize}

\newpage
\section{Solución a la dificultad de la pagina 43.}
\begin{lstlisting}[style=C]
for(int fase = 1; fase < npr; fase++){

  \* ---- Enviamos a destino/Recibimos desde fuente: p_anillo y a_anillo ----*\

   MPI_Sendrecv_replace(p_anillo, ncu, MPI_Coord, dst, 1, src, 1, MPI_COMM_WORLD, MPI_STATUS_IGNORE);
   MPI_Sendrecv_replace(a_anillo, ncu, MPI_Coord, dst, 1, src, 1, MPI_COMM_WORLD, MPI_STATUS_IGNORE);

  \* ---- Calculamos aceleraciones ---- *\

   calcularAceleracion(fase);

}
	
   \* ---- Enviamos a_anillo a destino y recibimos a_anillo de fuente ----*\

MPI_Sendrecv_replace(a_anillo, ncu, MPI_Coord, dst, 1, src, 1, MPI_COMM_WORLD, MPI_STATUS_IGNORE);

   \* ---- Sumamos aceleraciones ----*\

for(int i = 0; i < ncu; i++){
	a_local[i].x += a_anillo[i].x;
	a_local[i].y += a_anillo[i].y;
}
\end{lstlisting}
En el bucle de la línea 1 del código anterior, lo que se hace es que por cada fase, enviamos y recibimos \textit{p\_anillo} y \textit{a\_anillo}, es decir, cada proceso en su variable p\_anillo y a\_anillo está guardando las variables p\_anillo y a\_anillo de todos los demás, de este modo, todos los procesos tienen todos los datos. \\

Cuando termina el bucle, como hay que sumar todas las aceleraciones, en la última iteración para tener todas esas iteraciones, en la línea 16, se ejecuta la primitiva \textbf{\textit{MPI\_Sendrecv\_replace}} de a\_anillo, de modo que todos los procesos tienen las aceleraciones de todos los procesos. \\

Posteriormente, en las aceleraciones locales del proceso, se suma su propia aceleración local (a\_local[i]) más la aceleración del anillo (a\_anillo[i]) en las posiciones \textit{x} e \textit{y}.
\end{document}