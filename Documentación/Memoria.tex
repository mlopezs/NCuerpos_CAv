\documentclass[11pt]{article}
\usepackage{natbib}
\usepackage{url}
\usepackage[utf8]{inputenc} % Codificación
\usepackage{amsmath}
\usepackage{graphicx}
\graphicspath{{images/}} % Carpeta en la cual se van a buscar las imagenes
\usepackage{subfigure}	% Permite la Inclusión de subfiguras
%\usepackage{parskip} % Borrar identación de parrafos.
\setlength{\parskip}{3mm} % Longitud del espaciado entre parrafos
\usepackage[hidelinks]{hyperref} % Referencias (links)
\usepackage{fancyhdr}
\usepackage{vmargin}
\setpapersize{A4} % Formato del papel - A4
\setmarginsrb{3 cm}{2.5 cm}{3 cm}{2.5 cm}{1 cm}{1.5 cm}{1 cm}{1.5 cm} % Margenes
\usepackage{paralist} % Permite un mayor control sobre las listas
\usepackage{textcomp,marvosym,pifont} % Generación de símbolos especiales
\usepackage[usenames,dvipsnames,svgnames,x11names,table]{xcolor}

%%%%%%%%%%%%%%%%%%%%%%%%%%%%%%%%%%%%%%%%%%%%%%%%%%%%%%%%%%%%%%%%%%%%%%%%%%%%%%%%%%%%%%%%%
% Complemento para insertar código en la memoria:
% A partir de 'Listados de código cómodos y resultones con listings' de David Villa
% http://crysol.org/es/node/909

\usepackage{color}
\definecolor{gray97}{gray}{.97}
\definecolor{gray75}{gray}{.75}
\definecolor{gray45}{gray}{.45}

\usepackage{listings}
\lstset{ frame=Ltb,
	framerule=0pt,
	aboveskip=0.5cm,
	framextopmargin=3pt,
	framexbottommargin=3pt,
	framexleftmargin=0.4cm,
	framesep=0pt,
	rulesep=.4pt,
	backgroundcolor=\color{gray97},
	rulesepcolor=\color{black},
	texcl=true,
	%
	stringstyle=\ttfamily,
	showstringspaces = false,
	basicstyle=\small\ttfamily,
	commentstyle=\color{gray45},
	keywordstyle=\bfseries,
	%
	numbers=left,
	numbersep=15pt,
	numberstyle=\tiny,
	numberfirstline = false,
	breaklines=true,
}

% minimizar fragmentado de listados
\lstnewenvironment{listing}[1][]
{\lstset{#1}\pagebreak[0]}{\pagebreak[0]}

\lstdefinestyle{consola}
{basicstyle=\scriptsize\bf\ttfamily,
	backgroundcolor=\color{gray75},
}

\lstdefinestyle{C}
{language=C,
}
%%%%%%%%%%%%%%%%%%%%%%%%%%%%%%%%%%%%%%%%%%%%%%%%%%%%%%%%%%%%%%%%%%%%%%%%%%%%%%%%%%%%%%%%%

\usepackage{float} % Permite usar H en las figuras, de manera que se coloquen en la posición exacta en la que están en el código.


% Añade un comando para crear indicaciones de pulsación de teclas
\usepackage{tikz} % Paquete especializado en gráficos
\usetikzlibrary{shadows} % Necesario para poder crear nuevo comando de indicación de pulsación de tecla.
\newcommand*\tecla[1]{%   
	\tikz[baseline=(key.base)]
	\node[%
	draw,
	fill=white,
	drop shadow={shadow xshift=0.25ex,shadow yshift=-0.25ex,fill=black,opacity=0.75},
	rectangle,
	rounded corners=2pt,
	inner sep=1pt,
	line width=0.5pt,
	font=\scriptsize\sffamily
	](key) {#1\strut}
	;
}

\newif\ifspanish % Condicional que permite seleccionar el lenguage.
\spanishtrue


%%%%%%%%%%%%%%%%%%%%%%%%%%%%%%%%%%%%%%%%%%%%%%%%%%%%%%%%%%%%%%%%%%%%%%%%%%%%%%%%%%%%%%%%%
%%%%%%%%%					Principales variables del documento					%%%%%%%%%

\title{Práctica NCuerpos.}							% Titulo
\author{Marcos López Sobrino y Alberto Salas Seguín.}							% Autor
\date{\today}											% Fecha
\newcommand{\subject}{Computadores Avanzados.}						% Asignatura
\newcommand{\course}{4º Grado en Ingeniería Informática.}		% Curso

%\spanishfalse	% Descomentar esta línea si el trabajo está en inglés

%%%%%%%%%%%%%%%%%%%%%%%%%%%%%%%%%%%%%%%%%%%%%%%%%%%%%%%%%%%%%%%%%%%%%%%%%%%%%%%%%%%%%%%%%

\ifspanish
	\usepackage[spanish]{babel} % Paquete de español
	\newcommand{\dateText}{Fecha:}
	\renewcommand{\lstlistingname}{Listado} % Renombrar listados para que aparezcan en español.
	% Algoritmos
	\usepackage[ruled,vlined,spanish]{algorithm2e} % Permite pseudocódigos. NECESARIO INSTALAR texlive-science (sudo apt-get install texlive-science)
\else
	\usepackage[english]{babel} % Paquete de inglés
	\newcommand{\dateText}{Date:}
	% Algoritmos
	\usepackage[ruled,vlined,english]{algorithm2e}
\fi

\makeatletter
\let\thetitle\@title
\let\theauthor\@author
\let\thedate\@date
\makeatother

\pagestyle{fancy}
\fancyhf{}
\rhead{\theauthor}
\lhead{\thetitle}
\cfoot{\thepage}

\begin{document}
\begin{titlepage}
	\centering
    \includegraphics[scale = 0.25]{esilogo.png}\\[1.0 cm]	% Logo de la universidad
    \textsc{\LARGE Universidad de Castilla-La Mancha}\\[0.5 cm]	% Nombre de la universidad
    \textsc{\LARGE Escuela Superior de Informática}\\[2.0 cm]
	\textsc{\Large \textbf{\subject}}\\[0.5 cm]				% Asignatura
	\textsc{\large \course}\\[0.5 cm]						% Curso
	\rule{\linewidth}{0.2 mm} \\[0.4 cm]
	{ \huge \bfseries \thetitle}\\
	\rule{\linewidth}{0.2 mm} \\[1.5 cm]
	
	\begin{minipage}{0.4\textwidth}
		\begin{flushleft} \large
			\emph{Autor:}\\
			\textbf{\theauthor}
			\end{flushleft}
			\end{minipage}~
			\begin{minipage}{0.4\textwidth}
			\begin{flushright} \large
			\emph{\dateText} \\
			\thedate
		\end{flushright}
	\end{minipage}\\[1.5 cm]
 
	\vfill
	
\end{titlepage}

\tableofcontents
\newpage

\section{Instrucciones para la ejecución de los programas.}
Para la compilación y ejecución de ambos programas, se ha decidido realizar un \textit{Makefile}, donde para compilar tanto el programa paralelo rápido es suficiente con abrir un terminal e introducir el comando:
\begin{lstlisting}[style=C, numbers=none]
make all
\end{lstlisting}
Esta instrucción, como vemos a continuación, primero compila el programa y posteriormente lo ejecuta.
\begin{lstlisting}[style=C, numbers=none]
all: compile-par-rapido run-par-rapido
\end{lstlisting}

Donde \textbf{\textit{compile-par-rapido}} es:

\begin{lstlisting}[style=C, numbers=none]
compile-par-rapido:
	mpicc NCuerposParalelo_AlgoritmoRapido.c -o NCuerposParalelo_AlgoritmoRapido -lm -Wall
\end{lstlisting}

Y por su parte, \textbf{\textit{run-par-rapido}}
\begin{lstlisting}[style=C,numbers=none]
run-par-rapido:
	mpirun -np 2 NCuerposParalelo_AlgoritmoRapido
\end{lstlisting}


Siguiendo la transparencia 30 del documento, si solo interesa el tiempo el tiempo de ejecución, se ha incluido la directiva de compilación condicional \textbf{\textit{\# ifndef NO\_SAL}} en el programa, de modo que para únicamente obtener el tiempo de ejecución, en el terminal tenemos que introducir la instrucción \textbf{\textit{make compile-par-rapido-nosal}}, esta instrucción es la siguiente:

\begin{lstlisting}[style=C,numbers=none]
compile-par-rapido-nosal:
	mpicc NCuerposParalelo_AlgoritmoRapido.c -o NCuerposParalelo_AlgoritmoRapido -lm -Wall -D NO_SAL
\end{lstlisting}

Posteriormente, ejecutar con \textbf{\textit{make run-par-rapido}}.\\ 

Si en lugar de querer ejecutar el programa paralelo rápido, quisiésemos ejecutar el programa paralelo básico, sería suficiente con introducir el comando \textbf{\textit{make compile-par}} y \textbf{\textit{make run-par}}. Al igual que en el caso del algoritmo paralelo rápido, si solo nos interesa el tiempo de ejecución y no los resultados, lo que tenemos que hacer es compilar el programa mediante \textbf{\textit{make compile-par-nosal}} y después ejecutar con \textbf{\textit{make run-par}}.
\end{document}